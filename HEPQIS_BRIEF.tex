\section{Reviews}

\subsection{Quantum Machine Learning in High Energy Physics~\cite{Guan:2020bdl}}
\begin{itemize}
	\item \textbf{HEP Context: }Di-photon event classification, galaxy morphology classification, particle track reconstruction, and signal-background discrimination with the SUSY data set
	\item \textbf{Methods: }Quantum machine learning using quantum annealing, restrictive Boltzmann machines, quantum graph networks, and variational quantum circuits
	\item \textbf{Results and Conclusions: }This paper presents several papers on performing classification using quantum machine learning. The studies presented some of the challenges faced, such as the restrictive problem formulation for quantum annealers and the limited performance due to hardware restrictions for quantum-circuit-based machine learning.
\end{itemize}

\section{Whitepapers}

\subsection{Quantum Simulation for High Energy Physics~\cite{Bauer:2022hpo}}
\begin{itemize}
	\item \textbf{HEP Context: }TBD
	\item \textbf{Methods: }TBD
	\item \textbf{Results and Conclusions: }TBD
\end{itemize}\subsection{Quantum Computing for Data Analysis in High-Energy Physics~\cite{Delgado:2022tpc}}
\begin{itemize}
	\item \textbf{HEP Context: }Object reconstruction (tracking problem and  thrust for jet clustering), signal-background discrimination, detector simulations, and Monte Carlo event generation
	\item \textbf{Methods: }Amplitude amplification (generalization of Grover's algorithm), quantum annealing, hybrid quantum-classical neural networks, variational quantum circuits, quantum support vector machines, quantum convolutional neural networks, quantum variational autoencoders, and quantum generative models (quantum generative adversarial network and quantum circuit born machine)
	\item \textbf{Results and Conclusions: }\textbf{In object reconstruction:} ; \textbf{In classification:} the quantum implentation of the Combinatorial Kalman Filter based on amplitude amplification has a rigorous proof of quantum speedup, however; \textbf{In detector simulations and Monte Carlo event generation:} ; \textbf{Challenges and prospects:} 
\end{itemize}\subsection{Snowmass White Paper: Quantum Computing Systems and Software for High-energy Physics Research~\cite{Humble:2022vtm}}
\begin{itemize}
	\item \textbf{HEP Context: }TBD
	\item \textbf{Methods: }TBD
	\item \textbf{Results and Conclusions: }TBD
\end{itemize}

\section{Quantum Optimization Algorithms Based on Gate-Based Quantum Computers}

\subsection{Quantum Algorithms for Jet Clustering~\cite{Wei:2019rqy}}
\begin{itemize}
	\item \textbf{HEP Context: }Thrust, an event shape whose optimum corresponds to the most jet-like separating plane among a set of particles, focusing on the case of electron-positron collisions
	\item \textbf{Methods: }1) Created a quantum algorithm based on quantum annealing (enconded optimization problem as a QUBO problem); 2) Created quantum algorithm based on Grover search and describes two computing models, sequential model and parallel model, for loading classical data into quantum memory.
	\item \textbf{Results and Conclusions: }The overhead of data loading must be carefully considered when evaluating the potential for quantum speedups on classical datasets.
\end{itemize}\subsection{Quantum speedup for track reconstruction in particle accelerators~\cite{Magano:2021jzd}}
\begin{itemize}
	\item \textbf{HEP Context: }Track reconstruction
	\item \textbf{Methods: }TBD
	\item \textbf{Results and Conclusions: }This paper identifies the four fundamental routines in local track reconstruction methods: seeding, track building, cleaning, and selection. 
\end{itemize}

\section{Quantum Optimization Algorithms Based on Quantum Annealing}

\subsection{Quantum Algorithms for Jet Clustering~\cite{Wei:2019rqy}}
\begin{itemize}
	\item \textbf{HEP Context: }Thrust, an event shape whose optimum corresponds to the most jet-like separating plane among a set of particles, focusing on the case of electron-positron collisions
	\item \textbf{Methods: }1) Created a quantum algorithm based on quantum annealing (enconded optimization problem as a QUBO problem); 2) Created quantum algorithm based on Grover search and describes two computing models, sequential model and parallel model, for loading classical data into quantum memory.
	\item \textbf{Results and Conclusions: }The overhead of data loading must be carefully considered when evaluating the potential for quantum speedups on classical datasets.
\end{itemize}\subsection{Quantum Annealing for Jet Clustering with Thrust~\cite{Delgado:2022snu}}
\begin{itemize}
	\item \textbf{HEP Context: }TBD
	\item \textbf{Methods: }TBD
	\item \textbf{Results and Conclusions: }TBD
\end{itemize}\subsection{Degeneracy Engineering for Classical and Quantum Annealing: A Case Study of Sparse Linear Regression in Collider Physics~\cite{Anschuetz:2022rwu}}
\begin{itemize}
	\item \textbf{HEP Context: }TBD
	\item \textbf{Methods: }TBD
	\item \textbf{Results and Conclusions: }TBD
\end{itemize}\subsection{Leveraging Quantum Annealer to identify an Event-topology at High Energy Colliders~\cite{Kim:2021wrr}}
\begin{itemize}
	\item \textbf{HEP Context: }TBD
	\item \textbf{Methods: }TBD
	\item \textbf{Results and Conclusions: }TBD
\end{itemize}\subsection{Charged particle tracking with quantum annealing-inspired optimization~\cite{Zlokapa:2019tkn}}
\begin{itemize}
	\item \textbf{HEP Context: }TBD
	\item \textbf{Methods: }TBD
	\item \textbf{Results and Conclusions: }TBD
\end{itemize}\subsection{A pattern recognition algorithm for quantum annealers~\cite{Bapst:2019llh}}
\begin{itemize}
	\item \textbf{HEP Context: }Pattern recognition for track reconstruction using the TrackML dataset, relevant for analysis at the HL-LHC
	\item \textbf{Methods: }TBD
	\item \textbf{Results and Conclusions: }TBD
\end{itemize}\subsection{Adiabatic Quantum Algorithm for Multijet Clustering in High Energy Physics~\cite{Pires:2020urc}}
\begin{itemize}
	\item \textbf{HEP Context: }Jet clustering
	\item \textbf{Methods: }TBD
	\item \textbf{Results and Conclusions: }TBD
\end{itemize}

\section{Quantum Machine Learning Algorithms Based on Gate-Based Quantum Computers}

\subsection{Application of quantum machine learning using the quantum variational classifier method to high energy physics analysis at the LHC on IBM quantum computer simulator and hardware with 10 qubits~\cite{Wu:2020cye}}
\begin{itemize}
	\item \textbf{HEP Context: }Signal-background discrimination, where signal events are $t\bar{t}H$ ($H\rightarrow\gamma\gamma$) and $H\rightarrow\mu\mu$, and background events are dominant Standard Model processes
	\item \textbf{Methods: }Variational quantum circuits
	\item \textbf{Results and Conclusions: }TBD
\end{itemize}\subsection{Application of quantum machine learning using the quantum kernel algorithm on high energy physics analysis at the LHC~\cite{Wu:2021xsj}}
\begin{itemize}
	\item \textbf{HEP Context: }Signal-background discrimination, where signal events are $t\bar{t}H$ ($H\rightarrow\gamma\gamma$), and background events are dominant Standard Model processes
	\item \textbf{Methods: }Support vector machine with a quantum kernel estimator (QSVM-Kernel)
	\item \textbf{Results and Conclusions: }TBD
\end{itemize}\subsection{Application of Quantum Machine Learning to High Energy Physics Analysis at LHC Using Quantum Computer Simulators and Quantum Computer Hardware~\cite{Wu:2022tnc}}
\begin{itemize}
	\item \textbf{HEP Context: }Signal-background discrimination, where signal events are $t\bar{t}H$ ($H\rightarrow\gamma\gamma$), and background events are dominant Standard Model processes
	\item \textbf{Methods: }Variational Quantum Circuits (VQC), Quantum Support Vector Machine Kernel (QSVM-Kernel), and Quantum Neural Network (QNN)
	\item \textbf{Results and Conclusions: }TBD
\end{itemize}\subsection{Quantum Anomaly Detection for Collider Physics~\cite{Alvi:2022fkk}}
\begin{itemize}
	\item \textbf{HEP Context: }Anomaly detection in the four-lepton final state
	\item \textbf{Methods: }Variational Quantum Circuits (VQC) and Quantum Circuit Learning (QCL)
	\item \textbf{Results and Conclusions: }After comparing VQC and QCL to traditional classical machine learning algorithms, this paper states that there is no evidence that quantum machine learning provides any advantage to classical machine learning.
\end{itemize}\subsection{Event Classification with Quantum Machine Learning in High-Energy Physics~\cite{Terashi:2020wfi}}
\begin{itemize}
	\item \textbf{HEP Context: }Signal-background discrimination, where the signal is a SUSY process, in particular, a chargino-pair production via a Higgs boson, where the final state has two charged leptons and missing transverse momentum. The background event is a $W$ boson pair production $WW$ where each $W$ decays into a charged lepton and a neutrino.
	\item \textbf{Methods: }Variational Quantum Circuits (VQC) and Quantum Circuit Learning (QCL)
	\item \textbf{Results and Conclusions: }TBD
\end{itemize}\subsection{Quantum Machine Learning for $b$-jet identification~\cite{Gianelle:2022unu}}
\begin{itemize}
	\item \textbf{HEP Context: }TBD
	\item \textbf{Methods: }TBD
	\item \textbf{Results and Conclusions: }TBD
\end{itemize}\subsection{Classical versus Quantum: comparing Tensor Network-based Quantum Circuits on LHC data~\cite{Araz:2022haf}}
\begin{itemize}
	\item \textbf{HEP Context: }TBD
	\item \textbf{Methods: }TBD
	\item \textbf{Results and Conclusions: }TBD
\end{itemize}\subsection{Anomaly detection in high-energy physics using a quantum autoencoder~\cite{Ngairangbam:2021yma}}
\begin{itemize}
	\item \textbf{HEP Context: }TBD
	\item \textbf{Methods: }TBD
	\item \textbf{Results and Conclusions: }TBD
\end{itemize}\subsection{Implementation and analysis of quantum computing application to Higgs boson reconstruction at the large Hadron Collider~\cite{AlexiadesArmenakas:2021lrs}}
\begin{itemize}
	\item \textbf{HEP Context: }TBD
	\item \textbf{Methods: }TBD
	\item \textbf{Results and Conclusions: }TBD
\end{itemize}\subsection{Style-based quantum generative adversarial networks for Monte Carlo events~\cite{Bravo-Prieto:2021ehz}}
\begin{itemize}
	\item \textbf{HEP Context: }TBD
	\item \textbf{Methods: }TBD
	\item \textbf{Results and Conclusions: }TBD
\end{itemize}\subsection{Quantum convolutional neural networks for high energy physics data analysis~\cite{Chen:2020zkj}}
\begin{itemize}
	\item \textbf{HEP Context: }Classification of $\mu^+$, $e^-$, $\pi^+$, and $p$ at the Liquid Argon Time Projection Chamber (LArTPC) at Deep Undergroudn Neutrino Experiment (DUNE)
	\item \textbf{Methods: }Quantum Convolutional Neural Network (QCNN)
	\item \textbf{Results and Conclusions: }TBD
\end{itemize}\subsection{Hybrid Quantum-Classical Graph Convolutional Network~\cite{Chen:2021ouz}}
\begin{itemize}
	\item \textbf{HEP Context: }Classification of $\mu^+$, $e^-$, $\pi^+$, and $p$ at the Liquid Argon Time Projection Chamber (LArTPC) at Deep Undergroudn Neutrino Experiment (DUNE)
	\item \textbf{Methods: }Hybrid Quantum-Classical Graph Convolutional Neural Network (QGCNN)
	\item \textbf{Results and Conclusions: }TBD
\end{itemize}\subsection{Quantum Machine Learning for Particle Physics using a Variational Quantum Classifier~\cite{Blance:2020nhl}}
\begin{itemize}
	\item \textbf{HEP Context: }Signal-background discrimination, where the background is $pp \rightarrow t\bar{t}$ events, and the signal is $pp \rightarrow Z' \rightarrow t\bar{t}$ events
	\item \textbf{Methods: }Variational Quantum Classifier (VQC)
	\item \textbf{Results and Conclusions: }TBD
\end{itemize}\subsection{Application of a Quantum Search Algorithm to High- Energy Physics Data at the Large Hadron Collider~\cite{Armenakas:2020ymp}}
\begin{itemize}
	\item \textbf{HEP Context: }Detection of the exotic decays of Higgs boson used in Dark Sector searches ($H \rightarrow ZZ_d \rightarrow 4l$
	\item \textbf{Methods: }Grover's Algorithm
	\item \textbf{Results and Conclusions: }TBD
\end{itemize}\subsection{Quantum Support Vector Machines for Continuum Suppression in B Meson Decays~\cite{Heredge:2021vww}}
\begin{itemize}
	\item \textbf{HEP Context: }Signal-background classification, where signal is $B\bar{B}$ pair events, and background is $q\bar{q}$ pair events
	\item \textbf{Methods: }Quantum Support Vector Machine (QSVM)
	\item \textbf{Results and Conclusions: }TBD
\end{itemize}\subsection{Higgs analysis with quantum classifiers~\cite{Belis:2021zqi}}
\begin{itemize}
	\item \textbf{HEP Context: }Classification of $t\bar{t}H(b\bar{b}$ (signal) and $t\bar{t}b\bar{b}$ (background)
	\item \textbf{Methods: }Quantum Support Vector Machine (QSVM) and Variational Quantum Circuit (VQC)
	\item \textbf{Results and Conclusions: }TBD
\end{itemize}\subsection{Unsupervised Quantum Circuit Learning in High Energy Physics~\cite{Delgado:2022aty}}
\begin{itemize}
	\item \textbf{HEP Context: }TBD
	\item \textbf{Methods: }Quantum Circuit Born Machines (QCBM)
	\item \textbf{Results and Conclusions: }TBD
\end{itemize}

\section{Quantum Machine Learning Algorithms Based on Quantum Annealing}

\subsection{Solving a Higgs optimization problem with quantum annealing for machine learning~\cite{Mott:2017xdb}}
\begin{itemize}
	\item \textbf{HEP Context: }TBD
	\item \textbf{Methods: }TBD
	\item \textbf{Results and Conclusions: }TBD
\end{itemize}\subsection{Quantum adiabatic machine learning with zooming~\cite{Zlokapa:2019lvv}}
\begin{itemize}
	\item \textbf{HEP Context: }TBD
	\item \textbf{Methods: }TBD
	\item \textbf{Results and Conclusions: }TBD
\end{itemize}\subsection{Quantum algorithm for the classification of supersymmetric top quark events~\cite{Bargassa:2021jmk}}
\begin{itemize}
	\item \textbf{HEP Context: }TBD
	\item \textbf{Methods: }TBD
	\item \textbf{Results and Conclusions: }TBD
\end{itemize}

\section{Quantum Simulations}

\subsection{SU(2) hadrons on a quantum computer via a variational approach~\cite{Atas:2021ext}}
\begin{itemize}
	\item \textbf{HEP Context: }TBD
	\item \textbf{Methods: }TBD
	\item \textbf{Results and Conclusions: }TBD
\end{itemize}\subsection{Quantum Algorithm for High Energy Physics Simulations~\cite{Bauer:2019qxa}}
\begin{itemize}
	\item \textbf{HEP Context: }TBD
	\item \textbf{Methods: }TBD
	\item \textbf{Results and Conclusions: }TBD
\end{itemize}\subsection{Scalar Quantum Field Theories as a Benchmark for Near-Term Quantum Computers~\cite{Yeter-Aydeniz:2018mix}}
\begin{itemize}
	\item \textbf{HEP Context: }TBD
	\item \textbf{Methods: }TBD
	\item \textbf{Results and Conclusions: }TBD
\end{itemize}

\section{Quantum-Inspired Algorithms}

\subsection{Quantum-inspired event reconstruction with Tensor Networks: Matrix Product States~\cite{Araz:2021zwu}}
\begin{itemize}
	\item \textbf{HEP Context: }TBD
	\item \textbf{Methods: }TBD
	\item \textbf{Results and Conclusions: }TBD
\end{itemize}

\section{TBD}

\subsection{Unsupervised event classification with graphs on classical and photonic quantum computers~\cite{Blance:2020ktp}}
\begin{itemize}
	\item \textbf{HEP Context: }Anomaly detection, where background is $pp \rightarrow Z +$ jets events, and signal is $pp \rightarrow HZ$ events with subsequent decays $H \rightarrow A_1 A2$, $A_2 \rightarrow gg$, and $A_1 \rightarrow gg$, and the $Z$ boson decays leptonically to either $e$ or $\mu$
	\item \textbf{Methods: }TBD
	\item \textbf{Results and Conclusions: }TBD
\end{itemize}\subsection{Collider Events on a Quantum Computer~\cite{Gustafson:2022xwt}}
\begin{itemize}
	\item \textbf{HEP Context: }Parton shower algorithms
	\item \textbf{Methods: }TBD
	\item \textbf{Results and Conclusions: }TBD
\end{itemize}\subsection{Track clustering with a quantum annealer for primary vertex reconstruction at hadron colliders~\cite{Das:2019hrw}}
\begin{itemize}
	\item \textbf{HEP Context: }TBD
	\item \textbf{Methods: }TBD
	\item \textbf{Results and Conclusions: }TBD
\end{itemize}\subsection{Particle track classification using quantum associative memory~\cite{Quiroz:2020jmp}}
\begin{itemize}
	\item \textbf{HEP Context: }TBD
	\item \textbf{Methods: }TBD
	\item \textbf{Results and Conclusions: }TBD
\end{itemize}\subsection{Hybrid Quantum Classical Graph Neural Networks for Particle Track Reconstruction~\cite{Tuysuz:2021oai}}
\begin{itemize}
	\item \textbf{HEP Context: }TBD
	\item \textbf{Methods: }TBD
	\item \textbf{Results and Conclusions: }TBD
\end{itemize}\subsection{A Digital Quantum Algorithm for Jet Clustering in High-Energy Physics~\cite{Pires:2021fka}}
\begin{itemize}
	\item \textbf{HEP Context: }TBD
	\item \textbf{Methods: }TBD
	\item \textbf{Results and Conclusions: }TBD
\end{itemize}\subsection{A quantum algorithm for model independent searches for new physics~\cite{Matchev:2020wwx}}
\begin{itemize}
	\item \textbf{HEP Context: }TBD
	\item \textbf{Methods: }TBD
	\item \textbf{Results and Conclusions: }TBD
\end{itemize}\subsection{Dual-Parameterized Quantum Circuit GAN Model in High Energy Physics~\cite{Chang:2021ufg}}
\begin{itemize}
	\item \textbf{HEP Context: }TBD
	\item \textbf{Methods: }TBD
	\item \textbf{Results and Conclusions: }TBD
\end{itemize}\subsection{Running the Dual-PQC GAN on noisy simulators and real quantum hardware~\cite{Chang:2022dxc}}
\begin{itemize}
	\item \textbf{HEP Context: }TBD
	\item \textbf{Methods: }TBD
	\item \textbf{Results and Conclusions: }TBD
\end{itemize}\subsection{Quantum Generative Adversarial Networks in a Continuous-Variable Architecture to Simulate High Energy Physics Detectors~\cite{Chang:2021jne}}
\begin{itemize}
	\item \textbf{HEP Context: }TBD
	\item \textbf{Methods: }TBD
	\item \textbf{Results and Conclusions: }TBD
\end{itemize}\subsection{Quantum integration of elementary particle processes~\cite{Agliardi:2022ghn}}
\begin{itemize}
	\item \textbf{HEP Context: }TBD
	\item \textbf{Methods: }TBD
	\item \textbf{Results and Conclusions: }TBD
\end{itemize}\subsection{Quantum-inspired machine learning on high-energy physics data~\cite{Felser:2020mka}}
\begin{itemize}
	\item \textbf{HEP Context: }TBD
	\item \textbf{Methods: }TBD
	\item \textbf{Results and Conclusions: }TBD
\end{itemize}\subsection{Lattice renormalization of quantum simulations~\cite{Carena:2021ltu}}
\begin{itemize}
	\item \textbf{HEP Context: }TBD
	\item \textbf{Methods: }TBD
	\item \textbf{Results and Conclusions: }TBD
\end{itemize}

