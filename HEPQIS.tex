\documentclass[12pt,letterpaper]{article}
\usepackage{jheppub}

%\usepackage[hmargin=1.0in,vmargin=1.0in]{geometry}
%\usepackage{cite}
\usepackage[usenames,dvipsnames]{xcolor}	% For colors and names for color boxed links
% hyperref included through jheppub
\hypersetup{
	colorlinks=false,		% Surround the links by color frames (false) or colors the text of the links (true)
	citecolor=blue,		% Color of citation links
	filecolor=black,		% Color of file links
	linkcolor=red,		% Color of internal links (sections, pages, etc.)
	urlcolor=black,		% Color of url hyperlinks
	linkbordercolor=red, 	% Color of links to bibliography
	citebordercolor=blue,	% Color of file links
	urlbordercolor=blue	% Color of external links
}
% c.f.:
%	http://inspirehep.net/info/faq/general#utf8
%	https://tex.stackexchange.com/questions/172421/how-to-easily-use-utf-8-with-latex
%\usepackage{fontspec}

%%%%%%%%%%%%%%%%%%%%%%%%%%%%%%%
% Document body
%%%%%%%%%%%%%%%%%%%%%%%%%%%%%%%

\title{\boldmath A Living Review of Quantum Information \\ in High Energy Physics}

\abstract{Inspired by ``A Living Review of Machine Learning for Particle Physics"\footnote[1]{See \href{https://github.com/iml-wg/HEPML-LivingReview}{https://github.com/iml-wg/HEPML-LivingReview}.}, the goal of this document is to provide a nearly comprehensive list of citations for those developing and applying quantum information approaches to experimental, phenomenological, or theoretical analyses. Applications of quantum information science to high energy physics is a relatively new field of research. As a living document, it will be updated as often as possible with the relevant literature with the latest developments.  Suggestions are most welcome.}

\begin{document}
\maketitle

The purpose of this note is to collect references for quantum information science as applied to particle and nuclear physics.  The papers listed are in no particular order.  In order to be as useful as possible, this document will continually change. Please check back\footnote[2]{See \href{https://github.com/PamelaPajarillo/HEPQIS-LivingReview}{https://github.com/PamelaPajarillo/HEPQIS-LivingReview}.} regularly.  You can simply download the .bib file to get all of the latest references.  Suggestions are most welcome.

\input{HEPQIS_INPUTFILE}

\clearpage
\flushbottom
%%%%%%%%%%%%%%%%%%%%%%%%%%%%%%%
% References
%%%%%%%%%%%%%%%%%%%%%%%%%%%%%%%
%\bibliographystyle{uiuchept}
\bibliographystyle{JHEP}
\bibliography{HEPQIS}

\end{document}
