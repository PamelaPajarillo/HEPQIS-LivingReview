\documentclass[12pt,letterpaper]{article}
\usepackage{jheppub}

%\usepackage[hmargin=1.0in,vmargin=1.0in]{geometry}
%\usepackage{cite}
\usepackage[usenames,dvipsnames]{xcolor}	% For colors and names for color boxed links
% hyperref included through jheppub
\hypersetup{
	colorlinks=false,		% Surround the links by color frames (false) or colors the text of the links (true)
	citecolor=blue,		% Color of citation links
	filecolor=black,		% Color of file links
	linkcolor=red,		% Color of internal links (sections, pages, etc.)
	urlcolor=black,		% Color of url hyperlinks
	linkbordercolor=red, 	% Color of links to bibliography
	citebordercolor=blue,	% Color of file links
	urlbordercolor=blue	% Color of external links
}
% c.f.:
%	http://inspirehep.net/info/faq/general#utf8
%	https://tex.stackexchange.com/questions/172421/how-to-easily-use-utf-8-with-latex
%\usepackage{fontspec}

%%%%%%%%%%%%%%%%%%%%%%%%%%%%%%%
% Document body
%%%%%%%%%%%%%%%%%%%%%%%%%%%%%%%

\title{\boldmath A Living Review of Quantum Information \\ in High Energy Physics}

\abstract{
Inspired by "A Living Review of Machine Learning for Particle Physics"\footnote{See \href{https://github.com/iml-wg/HEPML-LivingReview}{https://github.com/iml-wg/HEPML-LivingReview}.}, the goal of this document is to provide a nearly comprehensive list of citations for those developing and applying quantum information approaches to experimental, phenomenological, or theoretical analyses.As a living document, it will be updated as often as possible to incorporate the latest developments.  A list of proper (unchanging) reviews can be found within.  Papers are grouped into a small set of topics to be as useful as possible.  Suggestions are most welcome.
}

\begin{document}
\maketitle

The purpose of this note is to collect references for quantum information science as applied to particle and nuclear physics.  A minimal number of categories is chosen in order to be as useful as possible.  Note that papers may be referenced in more than one category.  The fact that a paper is listed in this document does not endorse or validate its content - that is for the community (and for peer-review) to decide.  Furthermore, the classification here is a best attempt and may have flaws - please let us know if (a) we have missed a paper you think should be included, (b) a paper has been misclassified, or (c) a citation for a paper is not correct or if the journal information is now available.  In order to be as useful as possible, this document will continue to evolve so please check back\footnote{See \href{https://github.com/PamelaPajarillo/HEPQIS-LivingReview}{https://github.com/PamelaPajarillo/HEPQIS-LivingReview}.} before you write your next paper.  You can simply download the .bib file to get all of the latest references. 


\clearpage
\flushbottom
%%%%%%%%%%%%%%%%%%%%%%%%%%%%%%%
% References
%%%%%%%%%%%%%%%%%%%%%%%%%%%%%%%
%\bibliographystyle{uiuchept}
\bibliographystyle{JHEP}
\bibliography{HEPQIS}

\end{document}
